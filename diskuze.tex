\section*{Diskuze}
Fit závislosti $U(I_m)$ je na pohled velmi dobrý (viz graf \ref{g:vysledky}).

Námi změřená hodnota se přibližně shoduje s \SI{1.7e11}{\coulomb\per\kg} uvedenou v \cite{skripta}. Doporučená hodnota Národním institutem standardů a technologie z roku 2014 je přibližně \SI{1.76e11}{\coulomb\per\kg}.

Svazek měl nenulovou šířku a bylo obtížné zajistit, aby měl přesně žádoucí poloměr. Nejvíce se to projevilo při nízkých poloměrech a nízkých proudech. Projevilo se to tím, že jsme vždy měli určité rozpětí proudů $I_m$, u kterých jsme nebyli schopni rozlišit, kdy má svazek přesně kýžený poloměr. Tuto chybu jsme odhadli na \SI{0.02}{\ampere} kdykoliv $U<\SI{220}{\volt}$ a nebo se jedná o průměr \SI{40}{\mm}, v opačném případě \SI{0.01}{\ampere}. Ampérmetr měl chybu \SI{1}{\percent}. Po sečtení obou těchto chyb vyjde celková chyba přibližně jednotně pro všechna měření \SI{0.05}{\ampere}.   

Předpokládáme, že vzdálenost příček byla změřena dostatečně přesně, aby byla chyba zanedbatelná vzhledem k chybě způsobené nepřesným zaměřením svazku jako v předchozím odstavci.